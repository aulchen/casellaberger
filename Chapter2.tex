\documentclass{article}
\usepackage{amsmath}
\usepackage{amsfonts}
\usepackage{amssymb}

\newenvironment{proof}{\paragraph{Proof:}}{\hfill$\square$}
\newtheorem{theorem}{Theorem}
\newtheorem{lemma}[theorem]{Lemma}
\newtheorem{corollary}[theorem]{Corollary}

\newcommand{\diam}{\text{diam}}
\newcommand{\N}{\mathbb{N}}
\newcommand{\R}{\mathbb{R}}

\author{Arthur Chen}
\title{Casella Berger Chapter 2}
\date{\today}

\begin{document}
\maketitle

\section*{Problem 2.29}

\subsection*{Part c}

A nasty discrete distribution is the beta-binomial, with pmf

\[
P(X=k) = \frac{a}{k+a}
\frac{{n \choose k} {a+b-1 \choose a}}
{{n+a+b-1 \choose k+a}}
\]

where $n$, $a$, and $b$ are integers, and $y = 0, 1, 2 \dots n$. Use factorial moments to calculate the variance of the beta-binomial.

Note that being a probability distribution implies that

\[
\sum_{k=0}^n \frac{1}{k+a}
\frac{{n \choose k}}{{n+a+b-1 \choose k+a}}
=
[a {a+b-1 \choose a}]^{-1}
=
\frac{(a-1)!(b-1)!}{(a+b-1)!}
\]

Furthermore, expanding the binomial coefficients gives

\begin{align*}
\frac{1}{k+a}
\frac{{n \choose k}}{{n+a+b-1 \choose k+a}}
&=
\frac{\frac{n!}{(n-k)!k!}}
{
(k+a)
\frac{(n+a+b-1)!}{(n+b-k-1)!(k+a)!}
} \\
&=
\frac{(n+b-k-1)!(k+a-1)!}{(n-k)!k!}
\frac{n!}{(n+a+b-1)!}
\end{align*}

implying

\begin{align}
\sum_{k=0}^n \frac{(n+b-k-1)!(k+a-1)!}{(n-k)!k!}
&=\frac{(a-1)!(b-1)!(n+a+b-1)!}{n!(a+b-1)!} \\
&=(a-1)!(b-1)!{n+a+b-1 \choose a+b-1}
\end{align}

further implying that the probability mass function can be rewritten as

\begin{align}
P(Y=k) &=
\frac{n!(a+b-1)!}{(a-1)!(b-1)!(n+a+b-1)!} \frac{(n+b-k-1)!(k+a-1)!}{(n-k)!k!} \\
&= \alpha \frac{(n+b-k-1)!(k+a-1)!}{(n-k)!k!}
\end{align}

where $\alpha := \frac{n!(a+b-1)!}{(a-1)!(b-1)!(n+a+b-1)!}$ and doesn't depend on $k$. We first calculate the mean, $E(X)$.

\begin{align}
E(X) &= \alpha \sum_{k=0}^n
k \frac{(n+b-k-1)!(k+a-1)!}{(n-k)!k!} \\
\label{2_29KTermZero}
&= \alpha \sum_{k=1}^n
k \frac{(n+b-k-1)!(k+a-1)!}{(n-k)!k!} \\
&= \alpha \sum_{k=1}^n
\frac{(n+b-k-1)!(k+a-1)!}{(n-k)!(k-1)!}
\end{align}

where Equation \ref{2_29KTermZero} follows from the $k=0$ term being zero. 

\end{document}